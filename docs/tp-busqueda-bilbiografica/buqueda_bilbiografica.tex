\documentclass[11pt,a4paper,twoside]{thesis}

\usepackage{graphicx}
\usepackage[utf8]{inputenc}
\usepackage[spanish]{babel}
\usepackage[left=3cm,right=3cm,bottom=3.5cm,top=3.5cm]{geometry}
\usepackage{titlesec}
\usepackage{listings}
\usepackage{xcolor}
\usepackage{hyperref}
\usepackage{amsmath}
\usepackage{amssymb}
\usepackage[utf8]{inputenc}
\usepackage{enumitem}
\usepackage{booktabs}

\begin{document}

%%%% CARATULA

\def\autor{Adrian Norberto Marino}
\def\tituloTesis{Trabajo práctico 2: Búsqueda bibliografica}
\def\runtitulo{Resumen}
\def\runtitle{Trabajo práctico 2: Búqueda bibliografica}

%\def\director{Obi-Wan Kenobi}
%\def\codirector{Master Yoda}

\def\lugar{Buenos Aires, 2023}

% -----------------------------------------------------------------------------
% Caratula,  Resumen, agradecimientos y dedicatoria.
% -----------------------------------------------------------------------------
\input{cover.tex}
%
\frontmatter
\pagestyle{empty}
% -----------------------------------------------------------------------------
%
%
%
%\cleardoublepage
\tableofcontents
%
%
\mainmatter
\pagestyle{headings}
%
%
%
%
% -----------------------------------------------------------------------------
% Contenido de la tesis
% -----------------------------------------------------------------------------

\setitemize{itemsep=0.5pt}

\chapter{Introducción}

Los sistemas de recomendación tienen como objetivo principal proporcionar a los
usuarios productos, promociones y contenidos relevantes a sus preferencias o
necesidades. Estos sistemas permiten a los usuarios encontrar de forma ágil y
eficiente lo que están buscando. Formalizando esta definición, podemos decir
que los sistemas de recomendación buscan ayudar a un usuario o grupo de
usuarios a descubrir ítems que se ajusten a sus preferencias, dado un conjunto
de ítems que puede ser extenso o un amplio espacio de búsqueda.

\begin{sloppypar}
	Este objetivo puede variar dependiendo de cada negocio: Si consideramos un
	\textit{e-commerce} de \textit{delivery} gastronómico, su propósito sería
	ofrecer a los clientes platos relevantes a un precio asequible y con un tiempo
	de entrega aceptable.
\end{sloppypar}

Si hablamos de un \textit{e-commerce} de productos, su objetivo consiste en
proporcionar a los usuarios aquellos productos que satisfacen sus necesidades,
a un precio que estén dispuestos a pagar. Además, se busca garantizar una
experiencia satisfactoria con los vendedores.

En el negocio de visualización de contenido (audio, video, texto, etc..), el
objetivo es acercar a sus usuarios contenido a fin a sus preferencias para
mejorar su experiencia en la plataforma.

El objetivo principal en todos los casos es mejorar la conversión. En el campo
del \textit{marketing}, se define la conversión como las acciones realizadas
por los usuarios que están alineadas con los objetivos de la empresa. Por
ejemplo, aumentar el volumen de compras en un \textit{e-commerce} de productos,
incrementar la cantidad de entregas mensuales en un \textit{e-commerce} de
\textit{delivery} gastronómico, aumentar las impresiones de publicidad en
aplicaciones de visualización de contenido, prolongar el tiempo de permanencia
en plataformas de \textit{streaming} de audio o video, entre otros. Existen
numerosos ejemplos en los que el objetivo común es mejorar la conversión y el
compromiso del usuario con la marca, es decir, el \textit{engagement}.

Desde un enfoque técnico, los sistemas de recomendación se utilizan para
predecir el grado de preferencia de un usuario con un artículo específico. Esto
se logra aplicando algoritmos de optimización que minimizan la diferencia entre
el grado de preferencia esperado y el grado de preferencia real del usuario.
También existen otros enfoques que utilizan medidas de distancia para
determinar este grado de preferencia. En secciones posteriores, exploraremos
estos conceptos con mayor detalle.

\clearpage
\section{Tipos de sistemas de recomendación}

A continuación, en la figura~\ref{fig:clasification}, se pueden observar las
diferentes categorías y sub-categorías de los sistemas de recomendación:

\begin{figure}[!htb]
	\centering
	\includegraphics[width=12cm]{./images/clasificacion-sis-rec.png}
	\caption{Clasificación de tipos de sistemas de recomendaciones.}
	\label{fig:clasification}
\end{figure}

\subsection{Basados en Popularidad}

Este tipo de sistema de recomendación utiliza alguna característica de
popularidad de los ítems en cuestión. Algunos ejemplos de estas características
podría ser la cantidad de vistas, la cantidad de compras o la cantidad de
comentarios positivos, o una combinación de ellas. Luego, estos sistemas buscan
los K elementos más populares. Si bien este tipo de enfoque proporciona buenos
resultados para nuevos usuarios, sus recomendaciones no tienen en cuenta las
preferencias individuales de cada usuario, ya que se basan en estadísticas
comunes a todos los usuarios. Por esta razón, a menudo no se consideran
sistemas de recomendación en sentido estricto. No obstante, siguen siendo
ampliamente utilizados debido a su capacidad para generar una alta tasa de
conversión, a pesar de la falta de personalización.

\subsection{Basados en Contenido}

Este tipo de sistema de recomendación necesita un trabajo previo de ingeniería
de \textit{features} sobre los ítems. Se busca definir cuales son los
\textit{features} mas significativos para la tarea en cuestión, y cual es el
grado de adecuación de cada ítems a los \textit{features} seleccionados.

Por otro lado, existen dos formas de utilizas estos modelos:

\begin{itemize}
	\item El usuario esta registrado pero no realizo interacciones con los ítems: Este
	      escenario se da con usuarios que se registraron pero aun no han calificado
	      ningún ítems. En estos casos muchas aplicaciones optan por realizar una
	      encuesta inicial. Esta encuentra se utiliza para consultar al usuario
	      \textit{features} relevantes que permitirán comenzar a realizar
	      recomendaciones. Por ejemplo, se podría presentan una lista de géneros,
	      actores, directores, etc. Luego, el usuario realiza una puntuación de acuerdo a
	      su preferencia. Estas encuestas se suelen realizar cada cierto tiempo, dado que
	      hay usuarios que tienen una frecuencia muy baja de interacción con el sistema.
	\item El usuario se registro hace un tiempo y tiene cierta frecuencia de
	      interacciones con los ítems: En esta caso, se utilizan las clasificaciones del
	      usuario para realizar las recomendaciones. Este enfoque suele tener mejores
	      resultados, ya que muchas veces (dependiendo del dominio de las
	      recomendaciones) el usuario no conoce inicialmente cuales son sus preferencias
	      o bien puede cambiar de parecer con el tiempo. En comparación, este enfoque es
	      mejor al enfoque anterior, dado que el modelo tiene un conocimiento mas
	      actualizado sobre las preferencias del usuario, permitiendo realizar
	      recomendaciones mas alineadas a sus gustos.
\end{itemize}

Ambos enfoques puede combinarse. Supongamos que el científico de datos
seleccionar los géneros como \textit{features}. Inicialmente, puede realizar
una encuesta para conocer el grado de preferencia del usuario por cada género.
Luego, el usuario comienza a calificar ítems y estos valores se pueden
actualizar utilizando alguna medida basada en las calificaciones. De esta
forma, se podrían utilizar ambos enfoques de manera complementaria.

Luego, teniendo interacciones de los usuarios, se puede definir el grado de
preferencia de los usuarios a cada \textit{feature} definido para los ítems.
Con esta información, es posible encontrar tanto ítems como usuarios similares
y realizar recomendaciones del tipo:

\begin{itemize}
	\item Para el \textit{Usuario A} el cual tiene ciertos niveles de preferencia por
	      cada \textit{features}, se pueden recomendar los \textit{Ítem X} e \textit{Ítem
		      Y}. El modelo puede inferir el grado de preferencia del \textit{Usuario A} para
	      cada item existente y luego ordenarlos. En este caso, el \textit{Ítem X} es de
	      mayor preferencia para el usuario que el \textit{Ítem Y}.

	\item Dado el \textit{Usuario A}, el cual tiene preferencia por el \textit{Ítem X},
	      también podría tener preferencia por el \textit{Ítem Y}, por ser muy cercano o
	      similar al \textit{Ítem X}.

	\item Dos \textit{Usuarios A y B} cercanos o similares, tendrán preferencias
	      similares. De esta forma es posible recomendar ítem consumidos por el
	      \textit{Usuario A} al \textit{Usuario B} y vise versa.
\end{itemize}

La principal desventaja de este enfoque, es la necesidad de realizar ingeniería
de \textit{features} para encontrar los \textit{features} que produzcan
recomendaciones relevantes al usuario. El modelo no encuentra estos
\textit{features} automáticamente, sino que deben ser definidos de antemano
manualmente. Se puede apreciar que esto introduce un sesgo al momento de
seleccionar los \textit{features} o construirlos en base a datos referentes a
los ítems. Como ventaja, si se encuentran los \textit{features} correctos se
pueden lograr muy buenos resultados.

\subsection{Basados en Filtrado Colaborativos}

Estos modelos, a diferencia de los modelos basados en contenido, no requirieren
ingeniería de \textit{features}, lo que hace muy simple su implementación, ya
que únicamente es necesario registrar las interacciones de los usuarios para
con los ítems. Luego, el propio modelo encuentra automáticamente los
\textit{features} mas relevantes dependiendo de la cantidad de columnas que se
especifiquen (dimensiones de un vector \textit{Embedding}). Ejemplos de
interacciones podrían ser:

\begin{itemize}
	\item El \textit{Usuario A} visualizo el \textit{Ítem X} el dia 2 de marzo de 2022.
	\item El \textit{Usuario A} compro el \textit{Ítem X} el dia 10 de marzo de 2022.
	\item El \textit{Usuario A} califico al \textit{Ítem X} con 5 puntos el dia 25 de
	      marzo de 2022.
\end{itemize}

Ambos tipo de modelos, basados en contenido y filtros colaborativos,
personalizan sus recomendaciones. Es decir, ajustan las recomendaciones a las
preferencias de cada usuario particular. Además, ambos permiten encontrar
usuarios e ítems similares y recomendar ítems entre usuarios similares.

Por otro lado, los modelos basados en filtros colaborativos, descubren un
espacio latente de soluciones sin necesidad de recolectar datos y definir
\textit{features} en forma manual, a diferencia de los modelos basados en
contenido. La selección o construcción manual de \textit{features} puede llevar
a una solución sesgada, ya que no esta basada en datos sino en el juicio
experto del científico de datos. Esto puede llevar a una selección subjetiva de
los \textit{features} que se aleje de la realidad, introduciendo un sesgo en la
predicción.

No todo son rosas con estos modelos, dado que sufren un problema llamado
\textit{Cold start} o arranque en frio. Los usuarios nuevos son aquellos que
aun no han realizado ninguna interacción con el sistema. Estos modelos no
podrán realizar recomendaciones a estos usuarios, dado que requieren un mínimo
de interacciones para comenzar a ofrecer recomendaciones con cierta precisión.

Además, existen otros problemas referidos al cambiar la cantidad de
interacciones de los usuarios. Si pensamos en una solución donde alimentamos al
modelo con una ventana de interacciones para los últimos N meses, tendremos las
siguiente situaciones:

\begin{itemize}
	\item Usuarios nuevos: Los usuarios nuevos no tendrán interacciones. Por lo tanto,
	      este modelo no podrá realizar ninguna recomendación. En general, se establece
	      un mínimo de interacciones para que el modelo pueda realizar recomendaciones de
	      forma acertada.
	\item Usuarios con pocas interacciones: Por otro lado, tenemos a los usuarios que
	      tienen una baja taza de interacciones con el sistema o aplicación. Por ejemplo,
	      en un \textit{e-commerce} de venta de productos, hay usuarios que compran con
	      mucha frecuencia y otros muy de vez en cuando. Estos últimos, en general
	      tendrán una baja taza de interacción pudiendo caer por debajo del umbral mínimo
	      que requiere el modelo. De esta forma, tendremos usuarios que quedarán fuera
	      del modelo actual.
	\item Usuarios con muchas interacciones: En este caso, el usuario tiene una gran
	      cantidad de interacciones con ítems. Para estos usuarios, el modelo podrá
	      ofrecer recomendaciones relevantes, ya que cuanto mas interacciones se tenga,
	      el modelo se ajusta con mas facilidad a sus preferencias. Por otro lado, esto
	      puede ser una gran desventaja, ya que se produce un efecto de túnel. Es decir,
	      el usuario obtiene recomendaciones muy ajustadas a sus preferencias, perdiendo
	      la capacidad de descubrir nuevos ítems que podrían ser relevantes. Por esta
	      cuestión se suelen mezclar tanto recomendaciones personalizadas como
	      no-personalizadas, para favorecer el descubrimiento de nuevos ítems.

\end{itemize}

\subsection{Categorías dentro de los modelos basados en filtros colaborativos}

Dentro de los sistemas de recomendación basados en filtros colaborativos,
tenemos dos sub-clasificaciones referidas a la forma en la que se realizan las
predicciones.

\subsubsection{Basados en Memoria}

Este tipo de modelos, como su nombre lo indica, mantiene sus datos en memoria.
Se recorren todos los datos (\textit{full scan}) cada vez que se necesita
realizar un inferencia o predicción (fijando un número de vecinos a comparar).
Un ejemplo de estos modelos es el algoritmo de k vecinos cercanos
(\textit{KNN}), el cual mantiene una matriz rala de distancias en memoria, la
cual se recorre completamente para comparar las distancias entre filas o
columnas, usando alguna medida de distancia como puede ser la \textit{distancia
	coseno}, \textit{coseno ajustada}, \textit{manhattan}, etc.. Para mitigar el
problema de búsqueda exhaustiva (\textit{full scan}), se puede utilizar una
memoria \textit{cache} y asi realizar estas búsquedas una única vez. Otro
problema es su limitación al tamaño máximo de la memoria con la que se cuenta,
es decir, que el tamaño de la matriz depende de la memoria máxima disponible.
Esto puede mitigarse utilizando implementaciones de matrices rala, las cuales
comprimen los datos en memoria guardando unicamente las celdas que tienen
datos. Además, es posible utilizar un memoria \textit{cache} que mantenga en
memoria las búsqueda mas frecuentes y baje a almacenamiento secundario las
menos frecuentes. Todos estos problemas de \textit{performance} y uso de
recursos se deben a que \textit{KNN} no reduce la dimensionalidad de los datos,
como si lo hacen varias implementaciones basadas en \textit{embeddings},
\textit{auto-encoder}, redes neuronales etc.., donde lo que se busca una
representación mas compacta de los ítems y usuarios sin perder información. Mas
allá de estos problemas, los resultados obtenidos por estos modelos no están
muy alejados de aquellos que se encuentran en el estado del arte. Puede
recomendarse su uso cuando tenemos un dominio reducido, dada su simplicidad.

\subsubsection{Basados en Modelos}

Algunos ejemplos de estos modelos son los clasificadores bayesianos, redes
neuronales, algoritmos genéticos, sistemas difusos y la técnica de
descomposición matricial (\textit{SVD}). Estos modelos en general buscan
directa o indirectamente reducir la dimensionalidad de los datos. De esta
forma, es posible utilizarlos en dominios con una gran cantidad de datos.

\clearpage
\subsection{Modelos Híbridos o Ensambles}

Lod modelos híbridos o ensambles, son aquellos modelos que combinan mas de una
técnica de recomendación. Comúnmente son utilizados para resolver el problema
de \textit{Cold start} o arranque en frio que sufren los modelos de
recomendación basados en filtros colaborativos.

Los modelos de recomendación colaborativos, no puede realizar recomendaciones a
usuarios que aun no han calificado ítems o que no han realizado ninguna
interacción con la aplicación en cuestión. Para solucionar esta problemática,
se utilizan modelos de recomendación que no dependan de las interacciones de
los usuarios, como pueden ser modelos basados en contenido.

El uso de ensambles de modelos puede solucionar este problema de \textit{Cold
	start} o arranque en frio, y al mismo tiempo, es posible realizar
recomendaciones de mayor calidad a aquellas resultado de la predicción de cada
modelo por separado.

Por otro lado, utilizar ensambles no asegura una mejora en las recomendaciones,
ya que es algo muy dependiente de los datos, de la heterogeneidad de los
resultados de los modelos y las técnicas de ensamblado que se utilicen.

\subsection{Estrategias de ensamble de modelos}

A continuación se definen las estrategias de ensamble mas comunes:

\subsubsection{Switching}

La técnica de \textit{switching} consiste en intercambiar los modelos de
recomendación según la cantidad actual de interacciones de cada usuario. Es
recomendable filtrar las interacciones dentro de un período de tiempo de N
horas, días o meses, para luego calcular el número de interacciones del
usuario. Un ejemplo sería:

\begin{itemize}
	\item Si el usuario tiene menos de N interacciones, se utiliza un recomendador por
	      popularidad.
	\item Si el usuario tiene entre 5 y 10 interacciones, se utiliza un recomendador
	      basado en contenido.
	\item Si el usuario tiene mas de 20 interacciones, se utiliza un recomendador basado
	      en filtros colaborativos.
\end{itemize}

El principal problema de este enfoque es el cambio abrupto en el patrón de las
recomendaciones al cambiar de un modelo de recomendación a otro, ya que la
calidad de las recomendaciones puede variar mucho de un recomendador a otro. En
el ejemplo anterior, el cambio del recomendador por popularidad al recomendador
basado en contenido supone una mejora en la calidad de las recomendaciones, ya
que a partir de ese momento se estarían personalizando los resultados. Sin
embargo, ¿qué sucede cuando un usuario disminuye su frecuencia de interacción
con la aplicación? En este caso, se cambiaría de modelos más personalizados a
menos personalizados. Para disminuir los efectos de este problema, se podría
aumentar la ventana de tiempo para filtrar la interacciones, ocasionando el
efecto contrario a utilizar un ensamble.

\subsubsection{Mixing}

Esta técnica combina los \textit{ratings} predichos por cada modelo de
recomendación para cada ítem recomendado. Normalmente, se realiza una
normalización dividiendo por el \textit{rating} más alto, lo que da a cada ítem
un \textit{score} entre $0$ y $1$, donde $1$ es la puntuación más alta. Otra
alternativa seria calcular el \textit{score} mediante la media, promedio pesado
o mediana de los \textit{ratings} predichos por cada modelo y para cada ítem.
Esta técnica puede ser más efectiva si se combina con la técnica de
\textit{switching}, que suaviza la transición entre modelos mezclando las
recomendaciones de dos modelos cuando se alcanza un rango de interacciones
preestablecido como frontera entre estos.

\subsubsection{Weighted}

La técnica de \textit{Weighted} en un caso particular de la técnica de
\textit{Mixing}. Se refiere al caso en que se realiza un promedio pesado de los
\textit{ratings} predichos por cada modelo, para el mismo ítem. El problema es
que los pesos se asignan manualmente, ya sea arbitrariamente, o por
conocimiento del dominio. Es decir, no existe ningún proceso para optimizar o
descubrir estos hiper parámetros guiado por datos.

\subsubsection{Regresión Lineal}

Esta técnica es similar al enfoque \textit{Weighted}, pero en este caso, si se
optimizan los pesos utilizando un modelo de regresión linear. El proceso consta
de los siguientes pasos:

\begin{enumerate}
	\item Se entrena cada modelo de recomendación por separado utilizan el mismo conjunto
	      de entrenamiento y evaluación.
	\item Con cada modelo se predice valor de \textit{rating} que cada usuario asignara a
	      cada ítem del conjunto de entrenamiento. De esta forma como resultado, se tiene
	      un nuevo conjunto de entrenamiento con las siguientes columnas: usuario, item,
	      los \textit{ratings} predichos por cada modelos como columnas, y finalmente el
	      \textit{rating} real realizado por el usuario. Lo mismo sucede con el conjunto
	      de evaluación.
	\item Finalmente, se entrena y evalúa una regresión lineal como los conjuntos
	      construidos en el paso anterior. A diferencia del enfoque \textit{Weighted}
	      ahora si los pesos se ajustan de acuerdo a los datos de entrenamiento.
\end{enumerate}

\subsubsection{Stacking}

El enfoque \textit{Stacking} es una generalización del enfoque de
\textit{Regresión Lineal}. Este consta de aplicar cualquier modelo (incluido
regresión lineal) para ajustar los pesos, pudiéndose aplicar cualquier modelos
lineal o no lineal como: redes neuronales, regresiones polinómicas, etc.

\subsubsection{Feature-Weighted Linear Stacking}

Esta técnica a grandes rasgos guarda similitud con los últimos 3 métodos
expuestos (\textit{Weighted}, \textit{Regresión Lineal}, \textit{Stacking}).
Consta de aplicar un función lineal similar a una regresión, donde cada modelo
se puede pensar como una componentes de la regresión, cada una multiplicada por
una meta función o función \textit{feature-weights}. Estas meta funciones
pueden ser pensadas como pesos guiados por reglas duras. Es decir, sus valores
están pesados por reglas estáticas definidas previamente. Esta reglas ayudan a
manejar el nivel de contribución que tiene cada modelo sobre el \textit{rating}
resultado de la predicción. Por ejemplo, podríamos utilizar reglas duras para
aumentar la contribución de un modelo basado en contenido cuando el usuario
tiene pocas interacciones o aumentar la contribución de un modelo de filtros
colaborativos cuando el número de interacciones del usuario aumenta. En
definitiva, es una formas de realizar lo que se llama \textit{blend} o mezcla
de las contribuciones de cada modelo al \textit{rating} resultado aplicando una
transición suave.

\subsubsection{K-Arm Bandit utilizando Thompson sampling}

La técnica propuesta se basa en el conocido problema de los bandidos multi
brazo o también llamado el problema de las maquinas traga monedas.

El problema consiste en que tenemos una cantidad K de maquinas traga monedas y
contamos con una cantidad limitadas de fichas. Se busca producir la mayor
cantidad de ganancias, y en el proceso descubrir cual es la probabilidad de
existo de cada maquina. Todo esto utilizando la menor cantidad de fichas
posible. En definitiva, se busca maximizar la ganancia descubriendo de forma
temprana las probabilidades de éxito de cada maquina.

Este es un método iterativo, donde en cada iteración, el método selecciona una
máquina, realiza una jugada y registra si se ganó o se perdió.

La selección de la máquina puede ser al azar, o estar guiada por una heurística
o método estadístico.

Cuando la selección es al azar, se dice que se está en modo o etapa de
exploración, es decir, se gastan fichas para acumular resultados de cada
máquina y así registrar una serie de éxitos y pérdidas en cada máquina. Cada
máquina se puede pensar como una variable binomial, es decir toma dos valores:
gano o perdió.

Luego, con esta sucesión de eventos, se estima una distribución beta para cada
máquina, y finalmente se toma una muestra de un valor de esa distribución. Este
valor sera el valor mas probable o la probabilidad de éxito de la máquina.

Ya realizada una primera etapa de exploración, en las próximas iteraciones se
puede hacer uso del conocimiento actual y seleccionar la máquina con mayor
probabilidad de éxito. A esta etapa se le llama explotación, ya que se está
explotando o haciendo uso del conocimiento aprendido con anterioridad.

El proceso realiza una mezcla de estas dos etapas de forma estocástica,
haciendo uso de la estrategia \textit{epsilon greedy} (mejorada). Esta
estrategia, elige de forma aleatoria entre ambas técnicas de selección, con una
probabilidad de exploración \textit{epsilon}. Esta probabilidad ira
disminuyendo a medida que aumenta el número de iteraciones. Esta es una mejora
que permite restringir la exploración. Tiene mucho sentido, ya que a medida que
se realizan más jugadas, hay una mejor comprensión del problema, y es menos
necesario realizar una selección al azar.

En conclusión, se busca determinar la probabilidad de éxito de cada máquina,
realizando la menor cantidad de jugadas posibles.

Este enfoque se utiliza en el ámbito de sistemas de recomendación para
determinar el modelo con mayor probabilidad de éxito para cada usuario. Es
decir, cuál es el modelo que tiene mayor probabilidad de realizar una
recomendación similar a la que realizaría el usuario. Esta medición de éxito
depende de la métrica a evaluar.

A diferencia de varios de los modelos de ensamble anteriormente expuestos,
donde se requiere realizar una misma predicción para todos los modelos, este en
foque se realiza la predicción en un único modelo, debido a que es una
estrategia de selección de modelos. Esto se traduce en un uso mas eficiente de
los recursos de CPU, GPU y memoria del sistema, ya que estos modelos en general
se ejecutan como servicios \textit{cloud} (en la nube) en \textit{AWS}
(\textit{Amazon Web Services}) o \textit{GCP} (\textit{Google Cloud Platform}),
donde se paga unicamente por el uso de recursos que realizan los servicios o
modelo en este caso.

%%%% BIBLIOGRAFÍA

% Establece el estilo de las referencias bibliográficas
% otago, plain, apa, ieee, IEEEtran, etc...
\bibliographystyle{IEEEtran}
\renewcommand{\bibname}{Referencias}
\bibliography{cites} % Especifica el nombre del archivo .bib sin la extensión .bib

\end{document}