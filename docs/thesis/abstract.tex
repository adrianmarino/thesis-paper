%\begin{center}
%\large \bf \runtitulo
%\end{center}
%\vspace{1cm}
\chapter*{\runtitulo}

\noindent Este trabajo cubre la comparación de sistemas de recomendación basados en filtros colaborativos. Es una explicación exhaustiva del funcionamiento e implementación de la batería de modelos de recomendación colaborativos mas utilizados como son: \textit{GMF}, \textit{Biased-GMF}, \textit{KNN Item Based}, \textit{KNN User Based}, \textit{DeepFM} y \textit{NN-FM}. Se pretende comparar todos los modelos, utilizando métricas especializadas como el promedio de la precisión (\textit{AP@K}) y la media del promedio de la precisión (\textit{mAP@k}), y otras menos especializada como la raíz del error cuadrático medio \textit{RMSE}. Todos los modelos se entrenaron utilizando el mismo \textit{dataset}, construido a partir de los \textit{datasets} \textit{TMDB} y \textit{Movie Lens}. A grande rasgos, se ha encontrado que no existe una diferencia sustancial en precisión para los modelos propuestos. Ademas se encontró que modelo basados en \textit{Deep Learning} obtiene resultados ligeramente superiores a modelos mas clásicos, como la familia de modelos \textit{KNN}.

\bigskip

\noindent\textbf{Palabras claves:} Sistemas de Recomendación, Basados en Filtro Colaborativos, Basados en Contenido, Modelos Híbridos, \textit{GMF}, \textit{KNN}, \textit{NN-FM}, \textit{DeepFM}, \textit{mAP@k}.
