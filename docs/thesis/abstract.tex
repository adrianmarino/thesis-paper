%\begin{center}
%\large \bf \runtitulo
%\end{center}
%\vspace{1cm}
\chapter*{\runtitulo}

\noindent Este trabajo cubre la comparación de sistemas de recomendación basados en filtros colaborativos. Es una explica exhaustiva del funcionamiento e implementación de la batería de modelos de recomendación colaborativos mas utilizados como son: GMF, Biased-GMF, KNN Item Based, KNN User Based, DeepFM y NN-FM. Luego se pretende comparar todos los modelos utilizando métricas especialidades para sistemas de recomendación como son AP@K y mAP@k y otros menos especializada como RMSE. A grande rasgos se ha descubierto que no existe una diferencia sustancial en precisión entre modelos entrenados con un dataset generado a partir de TMDB y Movie Lens. También se descubre que modelo basados en Deep Learning obtiene resultados ligeramente mejore que modelos mas clásicos com la familia de modelos KNN. 

\bigskip

\noindent\textbf{Palabras claves:} Sistemas de Recomendación basados Filtro Colaborativos, Comparativa de Sistemas de Recomendación, Sistemas de Recomendación Híbridos, On-Hot Encoding, Embeddings, GMF, Biased-GMF, KNN Item Based, KNN User Based, DeepFM, NN-FM, RMSE, AP@k, mAP@k, TMDB, imdb, Movie Lens, Pytorch, Keras, TensorFlow.
