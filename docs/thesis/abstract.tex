%\begin{center}
%\large \bf \runtitulo
%\end{center}
%\vspace{1cm}
\chapter*{\runtitulo}

\noindent Este trabajo cubre la comparación de sistemas de recomendación basados en filtros colaborativos. Es una explicación exhaustiva del funcionamiento e implementación de la batería de modelos de recomendación colaborativos mas utilizados como son: \textit{GMF}, \textit{Biased-GMF}, \textit{KNN Item Based}, \textit{KNN User Based}, \textit{DeepFM} y \textit{NN-FM}. Luego se pretende comparar todos los modelos utilizando métricas especialidades para sistemas de recomendación como son \textit{AP@K} y \textit{mAP@k} y otros menos especializada como \textit{RMSE}. A grande rasgos se ha descubierto que no existe una diferencia sustancial en precisión entre modelos entrenados con un \textit{dataset} generado a partir de \textit{TMDB} y \textit{Movie Lens}. También se descubre que modelo basados en \textit{Deep Learning} obtiene resultados ligeramente mejores que modelos mas clásicos com la familia de modelos \textit{KNN}. 

\bigskip

\noindent\textbf{Palabras claves:} Sistemas de Recomendación, Filtro Colaborativos, Modelos Híbridos, \textit{GMF}, \textit{KNN}, \textit{NN-FM}, \textit{DeepFM}, \textit{mAP@k}.
